%&../../.preamble
\endofdump

\usetikzlibrary{external}
\tikzset{external/system call={pdflatex --shell-escape --fmt=../.preamble --halt-on-error -jobname "\image" "\endofdump\texsource"}}
\tikzexternalize[prefix=tikz/]

\title{Matematica}
\author{Marini Mattia}
\date{Maggio 2025}

\begin{document}
\maketitle
\license{Ripetizioni Arianna}
\tableofcontents
\newpage
\section{Studio di funzione}
\begin{enumerate}
	\item \textit{Dominio}
	      \begin{itemize}
		      \item Trovo valori per i quali la funzione non ha senso
	      \end{itemize}
	\item \textit{Segno}
	      \begin{itemize}
		      \item Risolvo equazione associata
		      \item Trovo zeri e intervalli in cui la funzione è positiva o negativa
	      \end{itemize}
	\item \textit{Limiti}
	      \begin{itemize}
		      \item Vedo come funzione si comporta a $ \pm \infty $ $ \rightarrow \lim_{x \to \pm \infty} f(x) $:
		      \item Vedo come funzione si comporta ad estremi del dominio e in punti esculsi dal dominio
		      \item Trovo asintoti orizzontali, verticali ed obliqui
		            \begin{itemize}
			            \item Verticali: se $ \lim_{x \to x_0} f(x) = \pm \infty  $ dove $ x_0 $ è un punto escluso dal dominio
			            \item Orizzontali: se $ \lim_{x \to \pm\infty} f(x) = k $, dove $ k $ è numero finito
			            \item Obliqui: se $ \lim_{x \to \pm\infty} \frac{f(x)}{x} = m $ dove $ m $ è finino ed è il coefficiente angolare della retta asintoto. Il quozioente di tale retta è : $ \lim_{x \to \pm \infty} \left[f(x) - mx\right] $
		            \end{itemize}
	      \end{itemize}
	\item \textit{Derivata prima}
	      \begin{itemize}
		      \item se $ f'\left(x\right) > 0 $ allora $ f $ cresce
		      \item se $ f'\left(x\right) < 0 $ allora $ f $ decresce
		      \item se $ f'\left(x\right) = 0 $ allora ho 3 opzioni:
		            \begin{itemize}
			            \item Punto di minimo locale/assoluto
			            \item Punto di massimo locale/assoluto
			            \item Flesso a tangenza orizzontale
		            \end{itemize}
	      \end{itemize}
	\item \textit{Derivata seconda}
	      \begin{itemize}
		      \item se $ f''\left(x\right) > 0 $ allora $ f $ ha concavità verso l'alto
		      \item se $ f''\left(x\right) < $ allora $ f $ ha concavità verso il basso
		      \item se $ f''\left(x\right) = 0 $ allora non si può dire nulla
	      \end{itemize}
\end{enumerate}
\subsection{Esercizi}
\begin{esercizio}{Studio di funzione}
	Studia la seguente funzione:
	\[
		x ln\left(x^2 \right)
	\]
\end{esercizio}
\begin{tikzpicture}
	\begin{axis}[
			xmin=-4, xmax=4,
			ymin=-2,ymax=2,
			restrict y to domain = -4:4, domain=-4:4, width=0.9\textwidth, height=0.9\textwidth, grid=major, samples=200,  ylabel=$f(x)$, xlabel=$x$, legend entries={$ $}]
		\addplot[black, thick] {x * ln(x^2)};
		\node (m1)[blackdot, label={90:$ m_1 $}] at (-0.38, {(-0.38) * ln((-0.38)^2)}) {};
		\node (m2)[blackdot, label={90:$ m_2 $}] at (0.38, {0.38 * ln(0.38^2)}) {};
		%
		\node [whitedot] at (0, {0 * ln(0^2)}) {};
		%
		\node [blackdot, label={110:$ x_0 $}] at (-1, {(-1) * ln((-1)^2)}) {};
		\node [blackdot, label={110:$ x_1 $}] at (1, {1 * ln(1^2)}) {};
	\end{axis}
\end{tikzpicture}
\vskip3mm
\begin{itemize}
	\item $ f'\left(x\right) = ln\left(x^2 \right) + 2 $
	\item $ f''\left(x\right) = \frac{2}{x} $
	\item No asintoti
\end{itemize}


\begin{esercizio}{Studio di funzione}
	Studia la seguente funzione:
	\[
		\frac{x^2  + 1}{x}
	\]
\end{esercizio}

\begin{tikzpicture}
	\begin{axis}[
			xmin=-10, xmax=10,
			ymin=-8,ymax=8,
			restrict y to domain = -40:40, domain=-10:10, width=0.9\textwidth, height=0.9\textwidth, grid=major, samples=201,  ylabel=$f(x)$, xlabel=$x$, legend entries={$ $}, unbounded coords=jump]
		\addplot[black, thick] {(x^2 + 1)/x};
		\addplot[dashed] {x};
		\node [blackdot, label={90:$ M $}] at (-1, {((-1)^2 + 1)/(-1)}) {};
		\node [blackdot, label={-90:$ m $}] at (1, {(1^2 + 1)/1}) {};
		\draw [dashed](0,-8)--(0,8);
	\end{axis}
\end{tikzpicture}
\begin{itemize}
	\item $ f'\left(x\right) = \frac{x^2 -1}{x^2 } $
	\item $ f''\left(x\right) = \frac{2}{x^3} $
	\item Asintoto verticale $ x=0 $ e obliquo $ y = x $
\end{itemize}
\vskip3mm

\end{document}
